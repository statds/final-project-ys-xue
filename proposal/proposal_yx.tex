\documentclass{article}
\usepackage{epsfig,graphicx,latexsym,amsfonts,amssymb,amsmath,verbatim}
\usepackage{booktabs, array, caption, natbib}
\usepackage{longtable}
\usepackage{rotating}
\usepackage{float}
\usepackage{caption}
\usepackage{color}
\usepackage{setspace}
\usepackage[top = .6in, bottom = .6in, left = 1in, right = 1in]{geometry}
\makeatletter % make @ act like a letter
\@addtoreset{equation}{section}
\makeatother  % make @ act like a non-letter
\setlength{\parskip}{3mm}
\setlength{\parindent}{8mm}
\usepackage{tabulary}

\usepackage{url}

\newcommand{\proglang}[1]{\textsf{#1}}
\newcommand{\pkg}[1]{{\normalfont\fontseries{b}\selectfont #1}}
\setlength\parindent{0pt}
\linespread{1}
\date{}
\title{STAT 6494 Data Science Project Proposal}
\author{Yishu Xue}
\begin{document}
\maketitle
\section{Introduction}
The fast development in information technology made communication between people
worldwide easier than ever. These advances are always accompanied by challenges.
Huge amounts of spam emails and texts are sent everyday. While spam filtering 
technologies have been widely used by major email service providers such as 
Gmail and Outlook, its application to mobile SMS is less pervasive. The iPhone, 
for example, has an ``unprotected'' inbox. Anybody who knows your mobile phone
number or iCloud account can send you messages without being blocked. 

There are, however, third-party apps on both iOS and Android platforms
that provide spam message filters. What is there filters based on? What algorithms
do they use? Will these algorithms be accurate in terms of sensitivity and 
specificity? In this project, I aim to build different classification models
on an SMS Spam 
Collection\footnote{\url{https://www.kaggle.com/uciml/sms-spam-collection-dataset}},
compare their performances, and look for the best classification scheme.

\section{Data}
The dataset is open data from Kaggle. It contains one set of SMS messages in 
English of 5,572 messages, tagged according being ham (legitimate) or spam.
747 of them are spam, while the rest 4,825 are ham. 

Examples:

\begin{tabulary}{\textwidth}{|c|L|}\hline
Spam & Had your mobile 11 months or more? U R entitled to Update to 
the latest colour mobiles with camera for Free! Call The Mobile Update Co 
FREE on 08002986030\\ \hline
Ham & Oops, I'll let you know when my roommate's done. \\ \hline
\end{tabulary}

\section{Methods}
The most frequently used classification method for SMS/email spam detection is 
Naive Bayes, 
followed by Support Vector Machine, Ensemble methods, logistic regression
 and K-Nearest 
Neighbors\citep{cormack2008email}. 
I look forward to implementing these methods on the SMS Spam 
Collection dataset, and see what specific features that they identify, what 
spam successfully ``cheated'' the classifiers, etc. I'm also interested in the 
application of EM for multinomial mixture models in text clustering, i.e.,
regardless of the class labels, whether the algorithm could successfully 
assign the SMS messages to two clusters.

\bibliographystyle{chicago}
\bibliography{proposal_yx}
\end{document}

